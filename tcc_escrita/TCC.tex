% Options for packages loaded elsewhere
\PassOptionsToPackage{unicode}{hyperref}
\PassOptionsToPackage{hyphens}{url}
\PassOptionsToPackage{dvipsnames,svgnames,x11names}{xcolor}
%
\documentclass[
  letterpaper,
  DIV=11,
  numbers=noendperiod]{scrreprt}

\usepackage{amsmath,amssymb}
\usepackage{iftex}
\ifPDFTeX
  \usepackage[T1]{fontenc}
  \usepackage[utf8]{inputenc}
  \usepackage{textcomp} % provide euro and other symbols
\else % if luatex or xetex
  \usepackage{unicode-math}
  \defaultfontfeatures{Scale=MatchLowercase}
  \defaultfontfeatures[\rmfamily]{Ligatures=TeX,Scale=1}
\fi
\usepackage{lmodern}
\ifPDFTeX\else  
    % xetex/luatex font selection
  \setmonofont[Scale = 1]{Ubuntu Mono}
\fi
% Use upquote if available, for straight quotes in verbatim environments
\IfFileExists{upquote.sty}{\usepackage{upquote}}{}
\IfFileExists{microtype.sty}{% use microtype if available
  \usepackage[]{microtype}
  \UseMicrotypeSet[protrusion]{basicmath} % disable protrusion for tt fonts
}{}
\usepackage{xcolor}
\setlength{\emergencystretch}{3em} % prevent overfull lines
\setcounter{secnumdepth}{5}
% Make \paragraph and \subparagraph free-standing
\ifx\paragraph\undefined\else
  \let\oldparagraph\paragraph
  \renewcommand{\paragraph}[1]{\oldparagraph{#1}\mbox{}}
\fi
\ifx\subparagraph\undefined\else
  \let\oldsubparagraph\subparagraph
  \renewcommand{\subparagraph}[1]{\oldsubparagraph{#1}\mbox{}}
\fi


\providecommand{\tightlist}{%
  \setlength{\itemsep}{0pt}\setlength{\parskip}{0pt}}\usepackage{longtable,booktabs,array}
\usepackage{calc} % for calculating minipage widths
% Correct order of tables after \paragraph or \subparagraph
\usepackage{etoolbox}
\makeatletter
\patchcmd\longtable{\par}{\if@noskipsec\mbox{}\fi\par}{}{}
\makeatother
% Allow footnotes in longtable head/foot
\IfFileExists{footnotehyper.sty}{\usepackage{footnotehyper}}{\usepackage{footnote}}
\makesavenoteenv{longtable}
\usepackage{graphicx}
\makeatletter
\def\maxwidth{\ifdim\Gin@nat@width>\linewidth\linewidth\else\Gin@nat@width\fi}
\def\maxheight{\ifdim\Gin@nat@height>\textheight\textheight\else\Gin@nat@height\fi}
\makeatother
% Scale images if necessary, so that they will not overflow the page
% margins by default, and it is still possible to overwrite the defaults
% using explicit options in \includegraphics[width, height, ...]{}
\setkeys{Gin}{width=\maxwidth,height=\maxheight,keepaspectratio}
% Set default figure placement to htbp
\makeatletter
\def\fps@figure{htbp}
\makeatother
% definitions for citeproc citations
\NewDocumentCommand\citeproctext{}{}
\NewDocumentCommand\citeproc{mm}{%
  \begingroup\def\citeproctext{#2}\cite{#1}\endgroup}
\makeatletter
 % allow citations to break across lines
 \let\@cite@ofmt\@firstofone
 % avoid brackets around text for \cite:
 \def\@biblabel#1{}
 \def\@cite#1#2{{#1\if@tempswa , #2\fi}}
\makeatother
\newlength{\cslhangindent}
\setlength{\cslhangindent}{1.5em}
\newlength{\csllabelwidth}
\setlength{\csllabelwidth}{3em}
\newenvironment{CSLReferences}[2] % #1 hanging-indent, #2 entry-spacing
 {\begin{list}{}{%
  \setlength{\itemindent}{0pt}
  \setlength{\leftmargin}{0pt}
  \setlength{\parsep}{0pt}
  % turn on hanging indent if param 1 is 1
  \ifodd #1
   \setlength{\leftmargin}{\cslhangindent}
   \setlength{\itemindent}{-1\cslhangindent}
  \fi
  % set entry spacing
  \setlength{\itemsep}{#2\baselineskip}}}
 {\end{list}}
\usepackage{calc}
\newcommand{\CSLBlock}[1]{\hfill\break\parbox[t]{\linewidth}{\strut\ignorespaces#1\strut}}
\newcommand{\CSLLeftMargin}[1]{\parbox[t]{\csllabelwidth}{\strut#1\strut}}
\newcommand{\CSLRightInline}[1]{\parbox[t]{\linewidth - \csllabelwidth}{\strut#1\strut}}
\newcommand{\CSLIndent}[1]{\hspace{\cslhangindent}#1}

\usepackage{pdflscape}
\newcommand{\blandscape}{\begin{landscape}}
\newcommand{\elandscape}{\end{landscape}}
\KOMAoption{captions}{tableheading,figureheading}
\makeatletter
\@ifpackageloaded{caption}{}{\usepackage{caption}}
\AtBeginDocument{%
\ifdefined\contentsname
  \renewcommand*\contentsname{Índice}
\else
  \newcommand\contentsname{Índice}
\fi
\ifdefined\listfigurename
  \renewcommand*\listfigurename{Lista de Figuras}
\else
  \newcommand\listfigurename{Lista de Figuras}
\fi
\ifdefined\listtablename
  \renewcommand*\listtablename{Lista de Tabelas}
\else
  \newcommand\listtablename{Lista de Tabelas}
\fi
\ifdefined\figurename
  \renewcommand*\figurename{Figura}
\else
  \newcommand\figurename{Figura}
\fi
\ifdefined\tablename
  \renewcommand*\tablename{Tabela}
\else
  \newcommand\tablename{Tabela}
\fi
}
\@ifpackageloaded{float}{}{\usepackage{float}}
\floatstyle{ruled}
\@ifundefined{c@chapter}{\newfloat{codelisting}{h}{lop}}{\newfloat{codelisting}{h}{lop}[chapter]}
\floatname{codelisting}{Listagem}
\newcommand*\listoflistings{\listof{codelisting}{Lista de Listagens}}
\makeatother
\makeatletter
\makeatother
\makeatletter
\@ifpackageloaded{caption}{}{\usepackage{caption}}
\@ifpackageloaded{subcaption}{}{\usepackage{subcaption}}
\makeatother
\ifLuaTeX
\usepackage[bidi=basic]{babel}
\else
\usepackage[bidi=default]{babel}
\fi
\babelprovide[main,import]{brazilian}
% get rid of language-specific shorthands (see #6817):
\let\LanguageShortHands\languageshorthands
\def\languageshorthands#1{}
\ifLuaTeX
  \usepackage{selnolig}  % disable illegal ligatures
\fi
\usepackage{bookmark}

\IfFileExists{xurl.sty}{\usepackage{xurl}}{} % add URL line breaks if available
\urlstyle{same} % disable monospaced font for URLs
\hypersetup{
  pdfauthor={Gabriel de Jesus Pereira},
  pdflang={pt-br},
  colorlinks=true,
  linkcolor={yellow},
  filecolor={Maroon},
  citecolor={Blue},
  urlcolor={yellow},
  pdfcreator={LaTeX via pandoc}}

\title{\includegraphics[width=1in,height=\textheight]{includes/ufpb.png}

Escrever título (escolher no final)}
\usepackage{etoolbox}
\makeatletter
\providecommand{\subtitle}[1]{% add subtitle to \maketitle
  \apptocmd{\@title}{\par {\large #1 \par}}{}{}
}
\makeatother
\subtitle{Universidade Federal da Paraíba - CCEN}
\author{Gabriel de Jesus Pereira}
\date{10 de junho de 2024}

\begin{document}
\maketitle

\renewcommand*\contentsname{Índice}
{
\hypersetup{linkcolor=}
\setcounter{tocdepth}{2}
\tableofcontents
}
\chapter{Resumo}\label{resumo}

\chapter{Capítulo 1}\label{capuxedtulo-1}

-- Fazer antes da conclusão --

\section{Introdução}\label{introduuxe7uxe3o}

\section{Objetivos}\label{objetivos}

\subsection{Objetivo Geral}\label{objetivo-geral}

\subsection{Objetivos Específicos}\label{objetivos-especuxedficos}

\section{Organização do Trabalho}\label{organizauxe7uxe3o-do-trabalho}

\newpage

\chapter{Capítulo 2}\label{capuxedtulo-2}

-- Fazer depois da metodologia --

\section{Recursos Computacionais}\label{recursos-computacionais}

\subsection{Linguagem de Programação
R}\label{linguagem-de-programauxe7uxe3o-r}

\subsection{Quarto}\label{quarto}

\subsection{Linguagem de Programação
Python}\label{linguagem-de-programauxe7uxe3o-python}

\subsection{Web Scraping}\label{web-scraping}

\newpage

\chapter{Metodologia}\label{metodologia}

\section{Os dados e o procedimento adotado para sua
obtenção}\label{os-dados-e-o-procedimento-adotado-para-sua-obtenuxe7uxe3o}

\hfill\break

O Web scraping, também conhecido como extração de dados da web, é uma
técnica utilizada para o processo de coleta de dados estruturados da web
de maneira automatizada. É um processo que vem sendo constantemente
utilizado por instituições públicas e privadas para a construção de
produtos que utilizam algoritmos de aprendizagem de máquinas, observa
ofertas e discontos, faz análise de mercado ou monitoração de marcas.

Neste projeto, para fins de estudo e análise do mercado imobiliário, os
dados foram coletados por meio de extração de dados do site do Zap
Imóveis. O Zap Imóveis é um site do Grupo OLX que reúne ofertas do
mercado imobiliário e que funciona como uma plataforma dinâmica para
facilitar a conexão entre quem deseja alugar, comprar ou vender um
imóvel; podendo servir também para corretores ou outros profissionais do
setor de imóveis. Este projeto foi possível graças as informações que
foram coletadas do site do Zap Imóveis em dois diferentes períodos do
ano de 2023. O primeiro deles, as informações foram coletadas utilizando
variados pacotes para raspagem de dados e proxies rotativas da linguagem
de programação R, a fim de evitar ser bloqueado pelos mecanismos de
segurança do site. Na segunda etapa, os dados foram coletados empregando
a linguagem de programação Python com as bibliotecas Scrapy e Playwrite,
que serve para web crawling e web scraping, e o Playwrite que serve para
testes em aplicativos da web, mas que neste caso foi utilizado para
manejar páginas dinâmicas.

Desta forma, com a ideia de modelar o valor do imóvel e analisar o
mercado imobiliário, foram coletados aqueles variáveis que estavam
disponíveis no site do Zap Imóveis e que poderiam de alguma forma ser
significativas ao tentar explicar o valor do imóvel durante a sua
modelagem. Assim, no total foram coletatas 23 variáveis, das quais 8 são
quantitativas e 15 qualitativas nominais, sendo 13 de caráter
dicotômico. No entanto, nem todas essas variáveis foram coletadas
diretamente do Zap Imóveis, a latitude e longitude foram obtidas pela
geocodificação do endereço utilizando o pacote tidygeocoder da linguagem
de programação R. Portanto, temos as seguites variáveis:

\begin{itemize}
\item
  Valor do imóvel: esta é a variável dependente, aquela que será
  modelada e será o principal objeto de estudo deste trabalho;
\item
  Área: área do imóvel em \(m^2\);
\item
  Condomínio: valor pago pelo condomínio;
\item
  IPTU: imposto cobrado de quem tem um imóvel urbano;
\item
  Banheiro: quantidade de banheiros presentes na propriedade;
\item
  Vaga de estacionamento: quantidade total de vagas de estacionamento;
\item
  Quarto: quantidade de quartos no imóvel;
\item
  Latitude: posição horizontal medida em frações decimais de graus;
\item
  Longitude: posição vertical que, assim como a latitude, é medida em
  frações decimais de graus;
\item
  Tipo do imóvel: foram obtidos 7 tipos de imóveis, apartamentos, casas,
  casas comerciais, casas de condomínio, casas de vila, coberturas,
  lotes comerciais e de condomínio;
\item
  Endereço: nome do endereço do imóvel;
\item
  Variáveis dicotômicas que indicam se o imóvel tem ou não aquela
  característica (representado como 1 ou 0, respectivamente): área de
  serviço, academia, elevador, espaço gourmet, piscina, playground,
  portaria 24 horas, quadra de esporte, salão de festa, sauna, spa e
  varanda gourmet.
\end{itemize}

No entanto, devido a observações feitas durante o estudo, nem todas
essas variáveis foram utilizadas para a modelagem do valor dos imóveis,
seja por conter muitos valores valores ausentes ou por não ter se
mostrado significante para o que se desejava explicar. Ainda, como a
coleta destes dados foram feitas em dois momentos distintos, temos dois
bancos de dados, um com 29712 observações e o outro com 14956. Por fim,
essas duas bases de dados foram unidas e, para não correr o risco de
conter imóveis repetidos, aqueles que tinham o mesmo número de
identificação foram removidos.

\section{Descritiva dos dados}\label{descritiva-dos-dados}

\hfill\break

A análise exploratória de dados marca uma das primeiras etapas de
qualquer estudo que utiliza a estatística como uma de suas principais
ferramentas, pois permite encontrar padrões de comportamento no dados,
descobrir relações entre as variáveis estudadas. Dessa forma, a primeira
etapa desse estudo, após a coleta e organização dos dados obtidos do Zap
Imóveis, foi fazer uma descritiva dos dados. Essa etapa permitiu
encontrar padrões nos diferentes tipos de imóveis bem como o seu tipo
pode influenciar na características do imóvel, o que, por consequência,
pode afetar o seu valor. Assim, para identificar esses diferentes
comportamentos, foram criados gráficos e tabelas a fim de caracterizar
as relações das variáveis independentes com a dependente.

\section{Aprendizado Supervisionado e não
supervisionado}\label{aprendizado-supervisionado-e-nuxe3o-supervisionado}

\hfill\break

Na aprendizagem de máquinas, uma das estapas mais importantes é saber
qual técnica será utilizada para resolver um problema que se enquadra em
diferentes formas de aprendizado. Para isso, existem mais de uma forma
em que um algoritmo consegue utilizar os dados e explicar o que está
sendo modelado a partir deles. No entanto, a maioria dos problemas de
aprendizado de máquinas recais em dois casos mais conhecidos:
aprendizado supervisionado e não supervisionado.

\subsection{Aprendizado
supervisionado}\label{aprendizado-supervisionado}

\hfill\break

Suponha uma regressão logística. Sabemos que na regressão logistíca
temos um modelo com a seguinte forma
\(Y_i = f\left(X\right) + \epsilon\), em que \(Y_i\) assume 0 ou 1 para
classificar o que está sendo modelado e representa a variável
dependente, \(f\left(X\right)\) representa as variáveis independentes
que serão utilizadas para a modelagem e \(\epsilon\) representa o erro
da regressão. Dessa forma, podemos considerar o caso em que a regressão
logística tenta classificar pacientes que podem ou não estar com
diabetes. Para isso, utilizariamos variáveis significativas para a
classificação do estado de cada paciente. Esse exemplo é conhecido como
aprendizagem supervisionada. Na aprendizagem supervisionada, busca-se
aprender \(Y_i\) atráves de um exemplo. Nesse caso, as variáveis
dependentes podem ser interpretadas como o exemplo, as informações de
relações de pacientes que podem ter ou não diabetes, e o estado do
paciente pode ser interpretado como o que se deseja aprender. Este
processo é entendido como \emph{aprendizado por exemplo}, HASTIE
\emph{et al.} (2009). O aprendizado supervisionado pode aparecer em
casos de regressão linear, regressão logística, ou até mesmo em métodos
mais modernos, como GAM, boosting e máquina de vetores de suporte, JAMES
\emph{et al.} (2023).

\subsection{Aprendizado não
supervisionado}\label{aprendizado-nuxe3o-supervisionado}

\hfill\break

Por outro lado, o aprendizado não supervisionado aparece em situações
mais desafiadores, pois não há um exemplo para explicar aquilo que se
pretende explicar. Este processo é conhecido como \emph{aprendizado sem
exemplo}, HASTIE \emph{et al.} (2009). Dessa forma, no aprendizado não
supervisionado, tem-se uma amostra com N observações
\(\left(x_1, ..., x_N\right)\) de um vetor aleatório \(X\) com densidade
conjunta \(f\left(x\right)\) em que o objetivo é inferir propriedades da
densidade sem ajuda de exemplos para cada observação. Assim, como há uma
falta de uma variável resposta \(y_i\) para supervisionar a análise,
pode-se procurar entender a relação entre as variáveis ou as
observações, JAMES \emph{et al.} (2023). Por exemplo, uma das técnicas
mais aplicadas em problemas que envolvem o aprendizado supervisionado é
a análise de cluster, em que o objetivo é determinar, com base em
\(x_1, ..., x_n\), se as observações são caracterizadas em grupos
distintos. Esse é um dos métodos que poderiam ser aplicados, por
exemplo, na análise de crédito de clientes de um cartão de crédito,
tornando possível analisar o seu perfil e classificá-lo em diferentes
grupos para recomendar produtos especificos adequados ao seu perfil.

\section{Métodos de reamostragem}\label{muxe9todos-de-reamostragem}

\hfill\break

\section{Tunagem de hiperparâmetros na aprendizagem de
máquina}\label{tunagem-de-hiperparuxe2metros-na-aprendizagem-de-muxe1quina}

Na aprendizagem de máquina, uma das principais etapas é a tunagem do
hiperparâmetros, que se consiste em encontrar a melhor combinação de
hiperparâmetros de um modelo. O método de GridSearch é a técnica mais
comum para otimização de hiperparâmetros utilizada em aprendizagem de
máquina. Essa técnica funciona testando todas as combinações possíveis
definidas, utilizando uma métrica de avaliação para selecionar a
combinação com os melhores resultados.

\hfill\break

\section{Modelos baseados em
árvores}\label{modelos-baseados-em-uxe1rvores}

\newpage

\chapter{Capítulo 4}\label{capuxedtulo-4}

\section{Resultados}\label{resultados}

\newpage

\chapter{Capítulo 5}\label{capuxedtulo-5}

\section{Conclusão e Discussão}\label{conclusuxe3o-e-discussuxe3o}

\chapter{Referências}\label{referuxeancias}

\phantomsection\label{refs}
\begin{CSLReferences}{0}{1}
\bibitem[\citeproctext]{ref-hastie2009elements}
HASTIE, T. \emph{et al.} \textbf{The elements of statistical learning:
data mining, inference, and prediction}. {[}S.l.{]}: Springer, 2009. V.
2.

\bibitem[\citeproctext]{ref-james2023introduction}
JAMES, G. \emph{et al.} \textbf{An introduction to statistical learning:
With applications in python}. {[}S.l.{]}: Springer Nature, 2023.

\end{CSLReferences}



\end{document}
